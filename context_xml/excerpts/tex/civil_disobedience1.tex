I heartily accept the motto,---"That government is best which governs
least;" and I should like to see it acted up to more rapidly and
systematically.  Carried out, it finally amounts to this, which also I
believe,---"That government is best which governs not at all;" and when
men re prepared for it, that will be the kind of government which they
will have. Government is at best but an expedient; but most
governments are usually, and all governments are sometimes
inexpedient. The objections which have been brought against a standing
army, and they are many and weighty, and deserve to prevail, may also
at last be brought against a standing government. The standing army is
only an arm of the standing government. The government itself, which
is only the mode which the people have chosen to execute their will,
is equally liable to be abused and perverted before the people can act
through it. Witness the present Mexican war, the work of comparatively
a few individuals using the standing government as their tool; for, in
the outset, the people would not have consented to this measure.\par 


This American government,---what is it but a tradition, though a
recent one, endeavoring to transmit itself unimpaired to posterity,
but each instant losing some of its integrity? It has not the vitality
and force of a single living man; for a single man can bend it to his
will. It is a sort of wooden gun to the people themselves; and, if
ever they should use it in earnest as a real one against each other,
it swill surely split. But it is not the less necessary for this; for
the people must have some complicated machinery of other, and hear its
din, to satisfy that idea of government which they have. Governments
show thus how successfully men can be imposed on, even impose on
themselves, for their own advantage. It is excellent, we must all
allow; yet this government never of itself furthered any enterprise,
but by the alacrity with which it got out of its way. {\sl It} does
not keep the country free. {\sl It} does not settle the West. {\sl It}
does not educate.  The character inherent in the  American people has
done all that has been accomplished; and it would have done somewhat
more, if the government has not sometimes got in its way. For
government is an expedient by which men would fain succeed in letting
one another alone; and as has been said, when it is most expedient,
the governed are most let alone by it. Trade and commerce, if they
were not made of India rubber, would never manage to bounce over the
obstacles which legislators are continually putting in their way; and,
if one were to judge theses men wholly by the effects of their
actions, and not partly by their intentions, they would deserve to be
classed and punished with those mischievous persons who put
obstructions on the railroads.\par

But to speak practically and as a citizen, unlike those who call
themselves no-government men I ask for not at once no government, but
at once a better government. Let every man make known what kind of a
government would command his respect, and that will be one step toward
obtaining it.\par

After all, the practical reason why, when the power is once in the
hands of the people, a majority are permitted, and for a long period
continue, to rule, is not because they are most likely to be in the
right, nor because this seems fairest to the minority, but because
they are physically the strongest. But a government in which the
majority rule in all cases cannot be based on justice, even as far as
men understand it. Can there not be a government in which majorities
do not virtually decide right and wrong, but conscience?---in which
majorities decide only those questions to which the  full of
expediency is applicable? Must the citizen ever for a moment, or in
the least degree, resign his conscience to the legislator? Why has
every man a conscience, then? I think that we should be men first, and
subjects afterward. It is not desirable to cultivate a respect for the
law, so much as for the right. The only obligation which I have a
right to assume, is to do at any time what I think right. It is truly
enough said, that a corporation has no conscience; but a corporation
of conscientious men is a corporation with a conscience. Law never
made men a whit more just; and, by means of their respect for it, even
the well-disposed are daily made the agents of injustice. A common and
natural result of an undue respect for law is, that you may see a
files o soldiers, colonel, captain, corporal, privates, powder-monkeys
and all, marching in admirable order over hill and dale to the wars,
against their wills, aye, against their common sense and consciences,
which makes it very steep marching indeed, and produces a palpitation
of the heart. They have no doubt that it is a damnable business in
which they are concerned; they are all peaceably inclined. Now, what
are they? Men at all? or small moveable forts and magazines, at the
service of some unscrupulous man in power? Visit the Navy Yard, and
behold a marine, such a man as an American government can make, or
such as it can make a man with its black arts, a mere shadow and
reminiscence of humanity, a man laid out alive and standing, and
already, as one  may ay, buried under arms with funeral
accompaniments, though it may be\par

\setupnarrower[left=24pt, right=24pt]
\startnarrower[left, right]
"Not a drum was heard, nor a funeral note,
As his corse to the ramparts we hurried;
Not a soldier discharged his farewell shot
O'er the grave where our hero we buried."
\stopnarrower
\par

The mass of men serve the State thus, not as men mainly, but as
machines, with their bodies. they are the standing army, and the
militia, jailers, constables, {\sl posse comitatus,} \&c. In most
cases there is no free exercise whatever of the judgement or the moral
sense; but they put themselves on a level with wood and earth and
stones; and wooden men can perhaps be manufactured that will serve the
purpose as well. such command no more respect than men of straw, or a
lump of dirt.  They have the same sort of worth only as horse and
dogs. Yet such as these even are commonly esteemed good citizens.
Others, as most legislators, politicians, lawyers, ministers, and
office-holders, serve the State chiefly with their heads: and, as they
rarely make any moral distinctions, they are as likely to serve the
devil, without intending it, as God. A very few, as heroes, patriots,
martyrs, reformers in the great sense, and {\sl men,} serve the State
with their consciences also, and so necessarily resist it for the
most part; and they are commonly treated by it as enemies. A wise man
will only be useful as a man, and will not submit to the "clay," and
"stop a hole to keep the wind away," but leave that office to his dust,
at least:---\par


\setupnarrower[left=24pt, right=24pt]
\startnarrower[left, right]
"I am too high-born to be propertied,
To be a secondary at control,
Or useful serving-man and instrument
To any sovereign state throughout the world."
\stopnarrower
\par

He who give himself entirely to his fellow-men appears to them useless
and selfish; but he who gives himself partially to them is pronounced
a benefactor and philanthropist.\par

How does it become a man to behave toward this American government
to-day? I answer that he cannot without disgrace be associated wit it.
I cannot for an instant recognize that political organization as {\sl
my} government which is the {\sl slave's} government also.\par

All men recognize the right of revolution; that is, the right to
refuse allegiance to and to resist the government, when its tyranny or
its inefficiency are great and unendurable. But almost all say that
such is not the case now. But such was the case, they think, in the
Revolution of '75. If one were to tell me that this was a bad government
because it taxed certain foreign commodities brought to its ports, it
is most probable that I should not make an ado about it, for I can do
without them: all machines have their friction; and possibly this does
enough good to counterbalance the evil. At any rater, it is a great
evil to make a stir about it.  But when the friction comes to have its
machine, and oppression and robbery are organized, I say, let us not
have such a machine any longer. In other words, when a sixth of the
population of a nation which has undertaken to be the refuge of liberty
t are slaves, and a whole country is unjustly overrun and conquered by
a foreign army, and subject to military law, I think that it s not too
soon for hones men to rebel and revolutionize. What makes this duty
the more urgent is the fact, that the country so overrun is not our
won, but ours is the invading army.\par 

Paley, a common authority with man on moral questions, in his chapter
on the "Duty of Submission to Civil Government," resolves all civil
obligation into expediency; and he proceeds to say, "that so long as
the interest of the whole society requires it, that is, so long as the
established government cannot be resisted or changed without public
inconveniency, it is the will of God that the established government
be obeyed, and no longer." --"This principle being admitted, the
justice of every particular case of resistance is reduced to a
computation of the quantity of the danger and grievance on the one
side, and of the probability and expense of redressing it on the
other." Of this, he says, every man shall judge for himself. But Paley
appears never to have contemplated those cases to which the rule of
expediency does not apply, in which a people, as well as an
individual, must do justice, cost what it may. If I have unjustly
wrested a plank from a drowning man, I must restore it to him though I
drown myself. This, according to Paley, would be inconvenient.. But he
that would save his life, in such a case, shall lose it. This people
must cease to hold slaves, and to make war on Mexico, though it cost
them their existence as a people.\par

In their practice, nations agree with Paley; but does any one think
that Massachusetts does exactly what is right at the present
crisis?\par

\setupnarrower[left=24pt, right=24pt]
\startnarrower[left, right]
"A drab of state, a cloth-o'silver slut,
To have her train borne up, and her soul trail in the dirt."
\stopnarrower
\par

Practically speaking, the opponents to a reform in Massachusetts are
not a hundred thousand politicians at the South, but a hundred thousand
merchants and farmers here, who are more interested in commerce and
agriculture than they are in humanity, and are not prepared to do
justice to the slave and to Mexico, {\sl cost what it may.} I quarrel
not with far-off-foes, but with those who, near at home, cooperate
with , and do the bidding of those far away, and without whom the
latter would be harmless. We are accustomed to say, that the mass of
men are unprepared; but improvement is slow, because the few are not
materially wiser or better than the many. It is not so important that
many should be as good as you, as that there be some absolute goodness
somewhere; for that will leaven the whole lump. There are thousands
who are {\sl in opinion} opposed to slavery and to the war, who yet in
effect do nothing to put an end to them; who, esteeming themselves
children of Washington and  Franklin, sit down with their hands in
their pockets, and say that they know not what to do, and do nothing,
who even postpone the question of freedom to the question of
free-trade, and quietly read the prices-current along with the latest
advices from Mexico, after dinner, and it may be, fall asleep them
both. What is the price-current of an honest man and patriot to-day?
They hesitate, and they regret, and sometimes they petition; but they
do nothing in earnest and with effect. They will wait, well disposed,
for others to remedy the evil, that they may no longer have it to
regret. At most, they give only a cheap vote, and a feeble countenance
and God-speed, to the right, as it goes by them. there are nine
hundred and ninety-nine patrons of virtue to one virtuous man; but it
is easier to deal with the real possessor of a thing than with the
temporary guardian of it. \par

