I should not obtrude my affairs so much on the notice of my readers if
very particular inquiries had not been made by my townsmen concerning
my mode of life, which some would call impertinent, though they do
not appear to me at all impertinent, but considering the circumstances,
very natural and pertinent. Some have asked what I got to eat; if I did
not feel lonesome; if I was not afraid; and the like. Others have been
curious to learn what portion of my income I h devoted to charitable
purposes; and some, who have large families, how many poor children I
maintained. I will therefore ask those of my readers who feel not
particular interest in me to pardon me if I undertake to answer some of
these questions in this book. In most books, the {\sl I} or first
person, is omitted; in this it will be retained; that, in respect to
egotism, is the main difference. We commonly do not remember that it
is, after all, always the first person that is speaking. I should not
talk so much about myself if there were any body else whom I knew as
well. Unfortunately, I am confined to this theme by the narrowness
of my experience. Moreover, I, on my side, require of every writer,
first or last, a simply and sincere account of his own life, and not
merely what he has heard of other men's lives; some such account as
he would send to his kindred from a distant land; for if he has lived
sincerely, it must have been in a distant land to me. Perhaps these
pages are more particularly addressed to poor students. As for the
rest of my readers, they will accept such portions s apply to them. I
trust that none will stretch the seems in putting on the coat, for it
may do good service to him whom it fits.\par
